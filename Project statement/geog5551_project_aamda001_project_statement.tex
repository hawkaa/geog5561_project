\documentclass{article}

\usepackage{parskip}
\usepackage{titling}
\usepackage[backend=bibtex,
style=numeric
]{biblatex}
\addbibresource{bibliography}

% Document metadata
\title{
    GEOG 5551 Term Project: Project statement\\
    Using a geographical information system to minimize risk and maximize profitability for internet service providers
}
\author{
    Aamdal, Haakon\\
    \texttt{aamda001@umn.edu}
}


% Margins
% Margins
\topmargin=-0.45in
\evensidemargin=0in
\oddsidemargin=0in
\textwidth=6.5in
\textheight=9.5in
\headsep=0.25in
\setlength{\droptitle}{-4em} 
\begin{document}
\maketitle

\emph{According to the class lecturer, Dudley Bonsal, the project statement was originally planned to have a different deadline than the intake assignment. Since they now happen to be due the same day, I've simply provided a copy of the first paragraph of the intake assigment, which describes the nature of the project.}

\section*{Project statement}
\label{sec:Project statement}
According to a white paper published by Bell Labs \cite{Bell_Labs2013-st} on internet traffic growth, the end-user internet connection traffic demand is forecasted to increase by 3.7 from 2012 to 2017. Nearly all of the demand is forecasted to come from residental and business fixed internet connections. There's over 30 internet service providers (ISPs) in the United States \cite{noauthor_undated-uf}, resulting in a highly competitive market with low margins of profit. This project will aim to increase profit for an imaginary internet service provider (ISP) by using a geographical information system. The system proposed in this project will use existing data about the ISP's network and combine them with demographic data and other attributes of interest in order to minimize risk and maximize profit when extending the network for new customers.
\printbibliography

\end{document}
